
\newcommand{\COURSE}{Teamprojekt - Super Mario Motion}
\newcommand{\STUDENTA}{Silas Jirikovsky}
\newcommand{\STUDENTB}{Timo Barton}
\newcommand{\STUDENTC}{Sascha}



\newcommand\Leq{\stackrel{\mathclap{\normalfont\mbox{\scriptsize L'Hôpital}}}{=}}

%Template
\documentclass[a4paper]{scrartcl}
\usepackage[dvipsnames]{xcolor}
\usepackage[utf8]{inputenc}
\usepackage{listings}
\usepackage{amsmath}
\usepackage[ngerman]{babel}
\usepackage{geometry,forloop,fancyhdr,fancybox,lastpage}
\geometry{a4paper,left=3cm, right=3cm, top=3cm, bottom=3cm}

%Math
\usepackage{amsmath,amssymb,amsfonts,tabularx}

%Figures
\usepackage{graphicx,tikz,color,float}
\usetikzlibrary{shapes,trees, arrows}
\usepackage{ragged2e}


%Algorithms
\usetikzlibrary{arrows}
\usepackage{float}
\usepackage[ruled,linesnumbered]{algorithm2e}
\usepackage[utf8]{inputenc}
\usepackage[ngerman]{babel}
\usepackage{geometry,forloop,fancyhdr,fancybox,lastpage}
\geometry{a4paper,left=3cm, right=3cm, top=3cm, bottom=3cm}
\usepackage[utf8]{inputenc}
\usepackage{amsfonts}
\usepackage[T1]{fontenc}
\usepackage[babel]{microtype}
\usepackage{calligra}
\usepackage[ngerman]{babel}
\usepackage{graphicx}
\usepackage{mathtools}
\usepackage{ wasysym }
\setlength{\textwidth}{13.5cm}
\setlength{\parindent}{0 cm}
\usepackage{delarray}
\usepackage[utf8]{inputenc}
\usepackage{amsthm}
\usepackage{makeidx}
\usepackage{empheq}
\usepackage{graphicx}
\usepackage{lmodern}
\usepackage[shortlabels]{enumitem}
\usepackage{tasks}
\usepackage{cases}
\usepackage{hyperref}
\usepackage{tcolorbox}
\usepackage[ruled,linesnumbered]{algorithm2e}

%Kopf- und Fußzeile
\pagestyle {fancy}
\fancyhead[L]{\STUDENTA \\ \STUDENTB \\ \STUDENTC}
\fancyhead[C]{\COURSE}
\fancyhead[R]{\today}

\fancyfoot[L]{}
\fancyfoot[C]{}
\fancyfoot[R]{Seite \thepage}

%Formatierung der Überschrift, hier nichts ändern
\def\header#1#2{
  \begin{center}
    {\Large\textbf{Übungsblatt #1}}\\
    {(Abgabetermin #2)}
  \end{center}
}

%Definition der Punktetabelle, hier nichts ändern
\newcounter{punktelistectr}
\newcounter{punkte}
\newcommand{\punkteliste}[2]{%
  \setcounter{punkte}{#2}%
  \addtocounter{punkte}{-#1}%
  \stepcounter{punkte}%<-- also punkte = m-n+1 = Anzahl Spalten[1]
  \begin{center}%
  \begin{tabularx}{\linewidth}[]{@{}*{\thepunkte}{>{\centering\arraybackslash} X|}@{}>{\centering\arraybackslash}X}
      \forloop{punktelistectr}{#1}{\value{punktelistectr} < #2 } %
      {%
        \thepunktelistectr &
      }
      #2 &  $\Sigma$ \\
      \hline
      \forloop{punktelistectr}{#1}{\value{punktelistectr} < #2 } %
      {%
        &
      } &\\
      \forloop{punktelistectr}{#1}{\value{punktelistectr} < #2 } %
      {%
        &
      } &\\
    \end{tabularx}
  \end{center}
}
\lstset{
    basicstyle=\ttfamily,
    keywordstyle=\bfseries,
    breaklines=true,
    showstringspaces=false
    mathescape
}